\documentclass[11pt, a4paper, landscape]{article}

% Graphics, plotting, images:
\usepackage{graphicx}

% Tweaking text borders:
\usepackage[top=0.89cm, bottom=4cm, left=1.27cm, right=1.31cm]{geometry}
\setlength{\headsep}{0.6cm}

% Making hyperlinks:
\usepackage[colorlinks=true, urlcolor=hyperlinkblue, linkcolor=black]{hyperref}

% Making headers/footers:
\usepackage{fancyhdr}
\setlength{\headheight}{100pt}
\pagestyle{fancy}
\fancyhf{}
\renewcommand{\headrulewidth}{0pt}% Remove header rule
\fancyhead[LE,LO]{\includegraphics[width=35mm]{{#LOGO#}}\\\hrulefill\\~}
\fancyhead[CE,CO]{~\\\hrulefill\\~}
%\fancyhead[RE,RO]{ \nouppercase{\leftmark\ | Page \thepage\ of \pageref{LastPage}}\\\hrulefill\\ {\lightfont \fontsize{10pt}{10pt}\selectfont {#SAMPLE_NAME#} | {#PANEL#} | \today}}
\fancyhead[RE,RO]{ \nouppercase{\leftmark\ | Page \thepage\ of \pageref{LastPage}}\\\hrulefill\\ {\lightfont \fontsize{10pt}{10pt}\selectfont SAMPLENAME | PANEL | \today}}


% tables spanning multiple pages
\usepackage{longtable}
\usepackage{array}
\newcolumntype{L}{>{\centering\arraybackslash}m{2.3cm}}
\newcolumntype{W}{>{\arraybackslash}m{22cm}}

\usepackage{booktabs}

%font setup
%\usepackage[sfdefault, medium]{roboto}
\usepackage{helvet}
\renewcommand\familydefault{\sfdefault}
\usepackage[T1]{fontenc}
\newcommand{\lightfont}{\fontseries{l}\selectfont}
\newcommand{\thinfont}{\fontseries{t}\selectfont}
\newcommand{\mediumfont}{\fontseries{m}\selectfont}
\newcommand{\boldfont}{\fontseries{b}\selectfont}

%multiple columns
\usepackage{multicol}
\setlength\columnsep{5mm}

%colors
\usepackage[table]{xcolor}
\definecolor{gray1}{HTML}{EDEDED}
\definecolor{lightblue1}{HTML}{E0F1F5}
\definecolor{darkblue1}{HTML}{2B8196}
\definecolor{hyperlinkblue}{HTML}{1F759B}

%tables
\usepackage{tabularx}
\usepackage{array}
\renewcommand{\arraystretch}{1.5}

\let\oldlongtable\longtable
\let\endoldlongtable\endlongtable
\renewenvironment{longtable}{\rowcolors{2}{lightblue1}{white}\oldlongtable} {
\endoldlongtable} %sets color scheme for long tables

%last page (for page count)
\usepackage{lastpage}

%change of caption styles
\usepackage[font=small]{caption}
\captionsetup{justification=raggedright,singlelinecheck=false}
\captionsetup[table]{ labelfont=it,textfont={it}}
\captionsetup[figure]{labelfont={it}}

%change of chapter styles
\usepackage{titlesec}
\titleformat{\section}
  {\normalfont \fontsize{16pt}{16pt}\selectfont}{\thesection}{1em}{}

\begin{document}

\noindent
{\fontsize{16pt}{16pt}\selectfont \color{darkblue1}{\textbf{Multisample report}}}

\medskip
\noindent
{\lightfont \today}

%
\section{QC information}

\footnotesize
{{#QCTABLE#}}


\normalsize
{{#BAD_AMPLICON_TABLE#}}

\newpage
\section{Shared variants across all samples}

{{#GATKHC_SHARED#}}

{{#LF_SHARED#}}

\newpage
\renewcommand{\arraystretch}{1.4}
\section{Annotated variants}
\footnotesize

{\captionof{table}{Legend}
\noindent
\begin{longtable}[l]{l W}
\rowcolor{lightblue1}
CHROM & chromosome\\
POS & position\\
ID & variant identity\\
REF & reference base\\
ALT & alternative base (i.e., variant)\\
QUAL & a phred-scaled quality score for the assertion made in ALT. i.e. -10log10 prob(call in ALT is wrong)\\
DP & filtered read depth\\
AF & allele frequency\\
FS & FisherStrand (Phred-scaled p-value using Fisher's Exact Test to detect strand bias (the variation being seen on only the forward or only the reverse strand) in the reads.  More bias is indicative of false positive calls. Be wary of SNP with FS > 60.0 and an indel with FS > 200.0).\\
AD & Allele Depth\\
EFF[*].GENE & Affected gene ("EFF" comes from "Effect Fields" from the SNPeff algorithm)\\
SB & Strand Bias\\
DP4 & Number of 1) forward ref alleles; 2) reverse ref; 3) forward non-ref; 4) reverse non-ref alleles, used in variant calling. Sum can be smaller than DP because low-quality bases are not counted.\\
\end{longtable}
{
\addtocounter{table}{-1}}}

{{#VCF_TABLES#}}

%
%
%\newpage
%\newgeometry{left=1.23cm, bottom=1.57cm, right=1.27cm, top=3.74cm}
%\setlength{\headsep}{1cm}
%
%\begin{landscape}
%{\color{darkblue1} \fontsize{14pt}{14pt}\selectfont Per amplicon average coverage}
%\begin{figure}[h]
%    \centering
%    \includegraphics[width=0.99\textwidth]{{#IMAGE2#}}
%    \caption{Average coverage (average number of sequencing reads
%    covering amplicon bases) is plotted for
%    each of the tested amplicons. Coverage at the 5\%, 10\%, 20\% and 5x of the mean sample coverage
%    is marked with the horizontal lines.}
%\end{figure}
%\end{landscape}
%
%\newpage
%\newgeometry{left=1.23cm, bottom=1.01cm, right=1.27cm, top=3.74cm}
%\setlength{\headsep}{0.5cm}
%
%\begin{landscape}

%\end{landscape}




\end{document}
